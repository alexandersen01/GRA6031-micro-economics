\documentclass[10pt]{article}

\usepackage{parskip}
\usepackage[margin=1in]{geometry} 
\usepackage{amsmath,amsthm,amssymb, graphicx, multicol, array}
\usepackage{enumitem}
\usepackage{amssymb}
\usepackage{href-ul}
 
\newcommand{\N}{\mathbb{N}}
\newcommand{\Z}{\mathbb{Z}}
 
\newenvironment{problem}[2][Problem]{\begin{trivlist}
\item[\hskip \labelsep {\bfseries #1}\hskip \labelsep {\bfseries #2.}]}{\end{trivlist}}

\begin{document}
 
\title{Lecture 1}
\author{Jakob Sverre Alexandersen\\
GRA4156 Accounting, Valuation and Financial Economics}
\maketitle

\tableofcontents
\newpage

\section{Bundling}

\begin{itemize}
    \item Different goods are sold in \textbf{one} package at one bundle price $P_B$
    \item One example is the office365 package, now THAT'S a bundle 
    \item P.B.: Pure Bundling – only the bundle is sold (ex: netflix)
    \item M.B.: Mixed bundling – consumers can choose between a bundle or each good separately (ex: a McD's meal)
    \item Note: $\Pi^{MB} \geq \Pi^{PB}$ because MB can always replicate PB by setting high prices for the separate goods
    \item Why it works: 
    \begin{itemize}
        \item Reduces variation in willingness to pay $\to$ setting ``only one price'' is less problematic!
        \item If negative correlatiuon in willingness to pay, then bundle!!!
    \end{itemize}
\end{itemize}

\newpage

\section{Game Theory}

In the majority of markets, firms interact with few competitors (oligopoly)
\begin{itemize}
    \item Need to consider rival's actions!
    \item Strategic interaction in prices, outputs, R\&D, advertising…
\end{itemize}

\hfill

Example of game from politics: 

You have 10 staff members to distribute to:
\begin{itemize}
    \item City A (10 votes)
    \item City B (6 votes)
    \item City C (4 votes)
\end{itemize}

\hfill

Where would you spend your 10 staff given the following payoffs: 

\[
\text{Opponent's staff} = 
\begin{cases}
    \text{more than you} & \text{you lose} \\
    \text{same as you} & \text{split the votes} \\
    \text{less than you} & \text{you win} \\
\end{cases}
\]

\hfill

\subsection{Overview}

\textbf{Key ideas and concepts in game theory}

\hfill 

Oligopoly models: 
\begin{itemize}
    \item Cournot (firms choose quantities simultaneously)
    \item Bertrand (firms choose prices simultaneously)
    \item Stackelberg (firms choose prices or quantities sequentially)
\end{itemize}

\hfill 

They are distinguished by:
\begin{itemize}
    \item Decision variable that firms choose (e.g. prices or quantities)
    \item Timing of the underlying game (i.e. simultaneous or sequential, finite or infinite)
\end{itemize}

\newpage

Use game theory to analyze situations with strategic interaction!
\begin{itemize}
    \item All players choose strategies (one for each player), which determine the action a player will take at any stage of the game
    \item the combination of strategies determines the outcome 
\end{itemize}

\hfill 

Hence, we have to define:
\begin{enumerate}
    \item the set of players: $i \in I$
    \item the strategies: $s_i \in S$
    \item the payoffs (utility): $u_i: S \to \mathbb{R}$
\end{enumerate}

\hfill 

Common knowledge / full rationality assumption: 
\begin{itemize}
    \item Each player is ``fully rational'' and each agent knows this
    \item Each player knows that each player knows the payoffs and strategies available to the players, and so on
\end{itemize}

\hfill 

Since players are rational:
\begin{itemize}
    \item No player will choose a \textit{dominated} strategy: A strategy is dominated when it gives lower return than other strategies, irrespective of what other players do
\end{itemize}

\hfill  

A strategy $s_i \in S$ is dominant for player $i$ if:
\begin{align*}
    u_i(s_i, s_{-i}) \geq u_i(s_i', s_{-i}) \quad\forall s_i' \in S_i \;\land\; \forall s_{-i} \in S_{-i}
\end{align*}

In plain text: ``if an action always gives a higher (lower) payoff compared to another action, whatever the other player does, we assume a player will (not) pick it''

\hfill 

\textbf{NOTE: all players know this, and we can rule out such strategies}

\hfill 

Definition (dominant strategy equilibrium)

A strategy profile $s^*$ is the dominant strategy equilibrium if for each player $i, s_i^*$ is a dominant strategy

\newpage

\subsection{Static vs. dynamic games and strategies}

\begin{itemize}
    \item Static: players choose actions simultaneously and play once (e.g., Cournot/Bertrand)
    \item Dynamic: players move sequentially or repeatedly (e.g., Stackelberg)
\end{itemize}

\hfill 

\textbf{Pure strategies}

A pure strategy is the choice by a player of a given action with certainty

Ex: play left or right with certainty

\hfill 

\textbf{Mixed strategies}

Mixed strategy is when a player might randomize between his actions 

Ex: a goalkeeper who plays left with $p = 0.3$ and right with $p = 0.7$

\hfill 

\textbf{Solution to firms' strategic interactions requires concept of equilibrium, first formalized by John F. Nash}
\begin{itemize}
    \item The market will be in equilibrium when no player wants to change its current strategy, given that no other players change their current strategy
    \item in this stable situation, players are best responding to each other: ``a strategy is a best response if it maximizes payoffs given the other player's strategy''
\end{itemize}

\hfill

A Nash equilibrium is a strategy profile $s^* \in S$ such that $\forall$ players $i \in I$

\begin{align*}
    u_i(s_i^*) \geq u_i(s_i', s_{-i}^*) \quad \forall s_i' \in S_i
\end{align*}

(i.e., no player can profitably deviate given the strategies of the other players)

\hfill 

\textbf{Example 1}

Two airlines compete in departure times (prices are set)

Imagine that 
\begin{itemize}
    \item 70\% if consumers prefer evening departure, 30\% prefer morning departure
    \item if airlines choose same departure times, they share the market equally
    \item pay-offs to the airlines as determined by market shares
\end{itemize}

\newpage

\textbf{what if there are no dominated or dominant strategies?}

\hfill 

Example: a pricing game between the two airlines 
\begin{itemize}
    \item 60 potential passengers with a reservation price of 500
    \item 120 additional passengers with a reservation price of 220
    \item assumption: price discrimination is not possible
    \\
    \item  costs are 200 per passenger no matter who is on the plane \\
    \item airlines must choose between a price of 500 and a price of 220
    \begin{itemize}
        \item equal prices $\to$ passengers are evenly shared
        \item different prices $\to$ cheapes airline gets \textit{all} the passengers 
    \end{itemize}
\end{itemize}


\section{Oligopoly}

\subsection{Oligopoly models}
\begin{itemize}
    \item Three main classes of oligopoly competition:
    \begin{itemize}
        \item Cournot 
        \item Bertrand
        \item Stackelberg and dynamic Hotelling
    \end{itemize}
    \item They are distinguished by
    \begin{itemize}
        \item The decision variable that firms can choose 
        \item the timing of the underlying game 
    \end{itemize}
    \item Concentrate on the Cournot first
\end{itemize}

\subsubsection{Cournot competition}

Assume: two firms making an identical product with demand for this product equal to:
\begin{align*}
    P = A - BQ = A - B(q_1 + q_2) \quad\text{($q_x$ is output firm $x$)}
\end{align*}

For each firm, $MC = c$ (i.e. constant marginal cost $c$ per unit)

\hfill 

What is the demand curve dacing each firm?
\begin{itemize}
    \item Treat the output of the other firm as constant
    \item For firm 2, demand is $P = (A - Bq_1) - Bq_2$
\end{itemize}

\begin{align*}
    MR_2 = (A - Bq_1) - 2Bq_2\\
    MR = MC \\
    A - Bq_1 - 2Bq_2 = c \\
    \to q^*_2 = \frac{(A - c)}{2B} - \frac{q_1}{2}
\end{align*}

\newpage

$q^*_2 = \frac{(A - c)}{2B} - \frac{q_1}{2} \quad$ this is the reaction function for firm 2
\begin{itemize}
    \item firm 2's profit-maximizing output for any output choice by firm 1
\end{itemize}

\hfill 

By exactly the same argument, the reaction function for firm 1 is: 

\begin{align*}
    q^*_1 = \frac{(A - c)}{2B} - \frac{q_2}{2}
\end{align*}

\textbf{Cournot-Nash equilibrium requires that both firms be on their reaction functions!!!}

In terms of $P$, note that they are the same product, customers don't really care about who sells the product: $p_1 = p_2 = p$ 

\hfill 

\textbf{Firm 1's problem: }
\begin{itemize}
    \item max $\Pi = p \times q_1 - c \times q_1 = [A - B \times (q_1 + q_2)] \times q_1 - c \times q_1$
\end{itemize}

\begin{align*}
    FOC_{q_1}: p'_{q_1} \times q_1 + p - c = 0 \\
    -Bq_1 + A - B(q_1 + q_2) - c = 0 \\
    \text{Solve for } q_1 \\
    q_1 = \frac{A - c}{2B} - \frac{q_2}{2} = q_1(q_2)
\end{align*}

$q_1$ is thus the reaction function (best response function)

\end{document}